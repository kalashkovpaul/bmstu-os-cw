\section*{\large ВВЕДЕНИЕ}
\addcontentsline{toc}{section}{ВВЕДЕНИЕ}

Операционные системы, в основе которых лежит ядро Linux, являются одними из самых популярных ОС в мире: к апрелю 2023 года ими пользуются 42\% пользователей мобильных устройств, 1.2\% пользователей ПК \cite{linux-statistics-2023}, а также 40.23\% разработчиков \cite{linux-dev-stats-stackoverflow}. 

В администрировании UNIX-подобных ОС существует понятие root аккаунта, у которого есть доступ ко всем командам и файлам системы \cite{root-definition}.
Захват root-прав или других привилегированных учетных данных повышает шанс хакеров остаться  незамеченным в сети и получить доступ к важным системам и данным.

\textbf{Целью работы} является разработка загружаемого модуля ядра для защиты файла от изменения привилегированным пользователем.
Для достижения поставленной цели необходимо выполнить следующие задачи:
\begin{enumerate}[label=\arabic*)]
	\item провести анализ действий, защиту от которых нужно обеспечить;
	\item провести анализ структур и функций, предоставляющих возможность реализовать поставленную задачу;
	\item разработать алгоритмы и структуру загружаемого модуля ядра, обеспечивающего защиту выбранного файла от  изменения привилегированным пользователем, с учётом предотвращения работы функций write, rename, remove и операций изменения владельца файла и группы пользователей файла;
\end{enumerate}

