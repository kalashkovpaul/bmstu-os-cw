\section{\large Технологический раздел}

\subsection{Выбор языка и среды программирования}
Как  основное  средство  реализации  и  разработки  ПО  был  выбран  язык
программирования  Cи \cite{c-lang}, поскольку в нём есть все инструменты для реализации загружаемого модудя ядра.
Средой программирования  послужил  графический  редактор  Visual  Studio  Code~\cite{vscode},  который известен  содержанием  большого  количество  плагинов,  ускоряющих  процесс разработки  программы.


\subsection{Реализация загружаемого модуля ядра}

В  расположенных  ниже  листингах \ref{code:1-1} -- \ref{code:1-5} приведена реализация загружаемого модуля ядра согласно разработанным алгоритмам, а на листинге \ref{code:makefile} --- Makefile для компиляции загружаемого модуля.

\clearpage

\begin{figure}[H]
\begin{code}
	\captionsetup{justification=centering}
	\captionof{listing}{Реализация загружаемого модуля ядра, часть 1}
	\label{code:1-1}
	\inputminted
	[
	frame=single,
	framerule=0.5pt,
%	framesep=20pt,
	fontsize=\small,
	tabsize=4,
	linenos,
	numbersep=5pt,
	xleftmargin=10pt,
	]
	{text}
	{code/module-1.c}
\end{code}
\end{figure}

\begin{figure}[H]
\begin{code}
	\captionsetup{justification=centering}
	\captionof{listing}{Реализация загружаемого модуля ядра, часть 2}
	\label{code:1-2}
	\inputminted
	[
	frame=single,
	framerule=0.5pt,
%	framesep=20pt,
	fontsize=\small,
	tabsize=4,
	linenos,
	numbersep=5pt,
	xleftmargin=10pt,
	]
	{text}
	{code/module-2.c}
\end{code}
\end{figure}

\begin{figure}[H]
\begin{code}
	\captionsetup{justification=centering}
	\captionof{listing}{Реализация загружаемого модуля ядра, часть 3}
	\label{code:1-3}
	\inputminted
	[
	frame=single,
	framerule=0.5pt,
%	framesep=20pt,
	fontsize=\small,
	tabsize=4,
	linenos,
	numbersep=5pt,
	xleftmargin=10pt,
	]
	{text}
	{code/module-3.c}
\end{code}
\end{figure}

\begin{figure}[H]
\begin{code}
	\captionsetup{justification=centering}
	\captionof{listing}{Реализация загружаемого модуля ядра, часть 4}
	\label{code:1-4}
	\inputminted
	[
	frame=single,
	framerule=0.5pt,
%	framesep=20pt,
	fontsize=\small,
	tabsize=4,
	linenos,
	numbersep=5pt,
	xleftmargin=10pt,
	]
	{text}
	{code/module-4.c}
\end{code}
\end{figure}

\begin{figure}[H]
\begin{code}
	\captionsetup{justification=centering}
	\captionof{listing}{Реализация загружаемого модуля ядра, часть 5}
	\label{code:1-5}
	\inputminted
	[
	frame=single,
	framerule=0.5pt,
%	framesep=20pt,
	fontsize=\small,
	tabsize=4,
	linenos,
	numbersep=5pt,
	xleftmargin=10pt,
	]
	{text}
	{code/module-5.c}
\end{code}
\end{figure}

\begin{figure}[H]
\begin{code}
	\captionsetup{justification=centering}
	\captionof{listing}{Makefile для компиляции загружаемого модуля}
	\label{code:makefile}
	\inputminted
	[
	frame=single,
	framerule=0.5pt,
%	framesep=20pt,
	fontsize=\small,
	tabsize=4,
	linenos,
	numbersep=5pt,
	xleftmargin=10pt,
	]
	{text}
	{code/Makefile}
\end{code}
\end{figure}



\subsection*{Вывод}
В данном разделе были рассмотрены  средства  разработки  программного обеспечения и приведены листинги исходного кода загружаемого модуля ядра.
